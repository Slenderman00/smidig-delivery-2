% !TeX root = Report.tex

\author{
    Joar Heimonen\\
    \texttt{contact@joar.me}
    \and
    Iselin Skorpen
    \and
    Salim Said
    \and
    Mostafa Mohammadi
    \and
    Ibrar Hussain
    \and
    Hassan Ali Bokhari
}
\documentclass[12pt]{article}
% include enumitem
\usepackage{enumitem}
\usepackage{listings}
\usepackage{sectsty}
\usepackage{color}
\usepackage{float}
\restylefloat{table}
\usepackage{graphicx}
\usepackage{biblatex}

\usepackage{changepage}

\usepackage{xcolor}
\usepackage{listings}

\addbibresource{Library.bib}

\title{\textbf{Group Venus} \\ 2nd Delivery: Design Sprint}
\date{\today}

\graphicspath{ {./images/} }

\begin{document}

\subsectionfont{\fontsize{12}{14}\selectfont}

\maketitle

\tableofcontents

\pagebreak

\section{Sprint Goal}
We had the goal of making a functional application, with navigational options, that we could use as 
a base for the next sprint by the end of this week. \break
$\-$ The result of this week was a functional app, where you can navigate several pages. 
It is not the best looking, but is a good base to continue our work with. It has low cost and 

\section{Backlog}
This is the current backlog for thie project.
\begin{itemize}
    \item Text input component is a work in progress.
    \item Login modal missing integration with project.
    \item Hypothetical reports on what can go wrong and access control solutions.
    \item React routing.
    \item Create Image slideshow component.
\end{itemize}
We also had a password strenght checker planned as an optional task if we had the time.
% Insert picture of the list from scrumwise

\section{Time}
Here is a picture of our Time list, in addition to this as this we spent about 8 hours in various meetings within the group.
% insert picture of time list
% insert picture of burndwn graph

\section{Reflection}
This section will anwser the following questions:
\begin{itemize}
    \item Could you have done anything differently?
    \item What were you particularly satisfied with?
\end{itemize}

\subsection{Could you have done anything differently?}
Improving our team communication and collabaration could have improved our workflow a lot. Ensuring that everyone was on the 
same page when starting production of our application. We should've had a more refined sketch done beforehand to improve 
clarity and end goal.

\subsection{What were you particularly satisfied with?}
We achieved our goal of developing a functional application, which will serve as a foundation for the next week's sprint. 
This weeks work provides us with a solid, low cost base which we can utilize to its fullest to continue our development.

\end{document}


\subsection{Ice breaker}
We were asked to present ourself visually, see \textit{Figure \ref{fig:IB}}.
This included anwsering the following questions:
\begin{itemize}
    \item Name
    \item Your icebreaker
    \item Internal forecast
    \item Favourite icecream
    \item Earlier relevant experience
\end{itemize}
\begin{figure}[h]
    \begin{adjustwidth}{-1in}{-1in}
        \centering
        \includegraphics[scale=1]{icebreaker.png}
        \caption{A visual presentation of a team member}
        \label{fig:IB}
    \end{adjustwidth}
\end{figure}
\clearpage
\subsection{Expert interview and HMW\cite{WhatHowMight} questions}
The expert interviews allows us to learn about the background and context of a potential solution.
The goal of this exercise is to create \textbf{How Might We} questions.
The following is a selection of some of our most popular questions:
\begin{itemize}
    \item How might we present information in an effective and engaging way
    \item How might we make information easily editable
    \item How might we create a unique application
\end{itemize}
\subsection{Longterm goals}
We were tasked with writing long term goals, on what the state of the application would be in two years, see \textit{Figure \ref{fig:LO}}.
The following are some of our most popular long term goals. 
\begin{itemize}
    \item In two years our application will be an important tool used by lots of buisnesses.
    \item In two years our application will be the standard for imformative displays
    \item In two years our application will be bug free
\end{itemize}
\begin{figure}[h]
    \begin{adjustwidth}{-1in}{-1in}
        \centering
        \includegraphics[scale=0.3]{longterm.png}
        \caption{A set of longterm goals}
        \label{fig:LO}
    \end{adjustwidth}
\end{figure}
\clearpage
\subsection{Sprint questions}
Sprint questions are a set of assumptions about our application presented as questions.
The goal is to reflect on what we have to do to make this application a success.
The following are some of our most popular sprint questions:
\begin{itemize}
    \item Can we make an application that works with as many types of information displays as possible?
    \item Can we make an application where information can be modified quickly and easily?
    \item Can we make an application that is accessible for all?
\end{itemize}
\subsection{Map and area of focus}
We were tasked with finding out what problems are most pressing for achieving our goals.
This was done by creating a map of our \textbf{How Might We}\cite{WhatHowMight} questions.
\subsection{concept sketch}
Concept sketches were created based on the sprint questions from earlier tasks.
See \textit{Figure \ref{fig:SH}} for an example of one of the more popular concept sketches.
\begin{figure}[h]
    \begin{adjustwidth}{-1in}{-1in}
        \centering
        \includegraphics[scale=0.5]{show-it.png}
        \caption{An example of a concept sketch}
        \label{fig:SH}
    \end{adjustwidth}
\end{figure}
\clearpage
% \subsection{lightning criticism}
% lorem ipsum
% \subsection{User test flow}
% lorem ipsum
% \subsubsection{Individual worksheets}
% lorem ipsum
% \subsubsection{Voting}
% lorem ipsum
\subsection{Storyboard}
Creating a Storyboard\cite{Storyboard2024} was the last task of the sprint.
A Storyboard is visual representation of how the user moves trough the application.
The Storyboard is used as a base for the actual design created in Penpot\cite{PenpotDesignTool}

\section{Reflection}
This section will anwser the following questions:
\begin{itemize}
    \item Did you find answers to the sprint questions?
    \item Could you have done anything differently?
    \item What were you particularly satisfied with?
    \item What would you do differently if you were to conduct a similar sprint again?
\end{itemize}

\subsection{Did you find answers to the sprint questions?}
During the sprint we had a persistant problem of team members producing similar anwsers to many of the tasks.
While this might seem positive at first the group felt that this stagnated the storyboard proccess.

\subsection{Could you have done anything differently?}
If we were more aware of the tasks and their requirements ahead of time we could have spent more time
concentrating on the application instead of trying to understand the tasks. 

\subsection{What were you particularly satisfied with?}
We are especially satisfied with some of the concepts that were drawn, even though they omitted certain elements.
Lots of the sketches had the same pervasive ideas, on how the basic layout of application. 


\subsection{What would you do differently if you were to conduct a similar sprint again?}
A majority of the group felt that the sprint was a bit to rigid in the timing of the activities.
We felt that there was a bit too much stuff happening. This resulted in us forgetting
basic considerations in our design proccess. All in all it was a positive experience
but if we were to do it again we would taylor it to our groups needs.

\subsection{Notes}
The group decided to use Penpot\cite{PenpotDesignTool} instead of Figma\cite{FigmaCollaborativeInterface} since Penpot more or less has feature parity with Figma, 
and Penpot is free and open-source software. We strongly believe that the design and development of applications should be free from external commercial pressures.
\printbibliography
